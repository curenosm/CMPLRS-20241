\documentclass{article}
\usepackage[utf8]{inputenc}
\usepackage[spanish]{babel}
\usepackage{algorithm}
\usepackage{algpseudocode}
\usepackage{amsmath}
\usepackage{amsfonts}
\usepackage{amssymb}
\usepackage{amsthm}
\usepackage{bm}
\usepackage{float}
\usepackage{forest}
\usepackage{graphicx}
\usepackage{hyperref}
\usepackage{lipsum}
\usepackage{listings}
\usepackage{ragged2e}
\usepackage{url}
\usepackage{xcolor}
\usepackage[T1]{fontenc}
% \usepackage{mathrsfs}
%\usepackage{tikz}
\usepackage[top=1in, left=1.25in, right=1.25in, bottom=1in]{geometry}

\title{Examen 3, Compiladores}
\author{
Bernal Núñez Raúl \\
Cureño Sanchez Misael
}
\date{Fecha límite de entrega: Miercoles 08 de noviembre a las 23:59.}

\begin{document}
\maketitle
\begin{enumerate}
  \item (25 pts.) Dada la siguiente gramática:

  $$E \rightarrow (E + T) \quad | \quad \textit{id} $$
  $$T \rightarrow (T * F) \quad | \quad \textit{id} $$
  $$F \rightarrow \textit{id} $$

  \begin{enumerate}
      \item Construir el AFD y la tabla de análisis para un analizador LR(0).
      \begin{proof}[\textbf{Solución: }]
        \quad \\ \\
      \end{proof}
      
      \item Construir el AFD y la tabla de análisis para un analizador SLR(1).
      \begin{proof}[\textbf{Solución: }]
        \quad \\ \\
      \end{proof}
      
      \item la cadena $(6 + (4 + 2))$ con ambos analizadores.
      \begin{proof}[\textbf{Solución: }]
        \quad \\ \\
      \end{proof}
  \end{enumerate}

  \item (25 pts.) Indica si los siguientes enunciados son falsos o verdaderos. Justifica tu respuesta.
  \begin{enumerate}
      \item Una gramática LR(1) que no es LALR(1) debe tener sólo conflictos de reducción-reducción.
      \begin{proof}[\textbf{Solución: }]
        \quad \\ \\
      \end{proof}
      
      \item Pueden haber gramáticas SLR(1) que no sean LALR(1).
      \begin{proof}[\textbf{Solución: }]
        \quad \\ \\
      \end{proof}
  \end{enumerate}

  \item (25 pts) Dada la siguiente gramática con atributos:
  \begin{enumerate}
      \item Mostrar el árbol sintáctico (gramatical o ASA) decorado para la cadena de entrada \textit{aaaa} e indicar la salida que provoca el análisis semántico.
      \begin{proof}[\textbf{Solución: }]
        \quad \\ \\
      \end{proof}
  \end{enumerate}

  \item (25 pts) Dada la siguiente gramática que genera números binarios:

  $$ A \rightarrow AB \quad | \quad B$$
  $$ B \rightarrow 0 \quad | \quad 1$$

  \begin{enumerate}
      \item Obtener la gramática con atributos que al analizar un número binario calcule su equivalente en decimal.
      \begin{proof}[\textbf{Solución: }]
        \quad \\ \\
      \end{proof}
      
      \item Mostrar el árbol para la cadena de entrada 101 e indicar la salida que provoca el análisis semántico.
      \begin{proof}[\textbf{Solución: }]
        \quad \\ \\
      \end{proof}
  \end{enumerate}
\end{enumerate}

\end{document}